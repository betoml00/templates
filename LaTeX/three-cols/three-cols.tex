\documentclass[10pt, landscape]{article}
\usepackage[spanish, es-noshorthands]{babel}
\decimalpoint
\usepackage{amssymb,amsmath,amsthm,amsfonts}           % Paquetes de matemáticas
\usepackage{multicol,multirow}                         % Varias columnas
\usepackage[margin=1cm, top=1cm, landscape]{geometry}  % Márgenes
\usepackage[colorlinks=true,citecolor=blue,linkcolor=blue]{hyperref}
\usepackage{fancyhdr}                                  % Headers
\usepackage{centernot}

% ---- GRÁFICAS ----
\usepackage{tikz}
\usepackage{scalerel}
\usepackage{pict2e}
\usepackage{tkz-euclide}
\usetikzlibrary{calc}
\usetikzlibrary{patterns,arrows.meta}                   % Mejores flechas
\usetikzlibrary{shadows}                                % Sombras
\usetikzlibrary{external}                               % ¿?
\usetikzlibrary{decorations.pathreplacing,calligraphy}  % Para poder graficar el '{'
% Plots
\usepackage{pgfplots}
\pgfplotsset{compat=newest}                             % Ancho de las gráficas: width=6.5cm,
\usepgfplotslibrary{statistics}
\usepgfplotslibrary{fillbetween}

% ---- COLORES ----
\usepackage{xcolor}
\definecolor{miAzul}{RGB}{13, 33, 161}
\definecolor{miRojo}{RGB}{186, 24, 27}

% ---- FORMATO DE TÍTULOS ----
\usepackage{titlesec}
\titleformat{\section}
{\normalfont\large\bfseries}{}{0mm}{}
\titleformat{\subsection}
{\normalfont\normalsize\bfseries}{}{0mm}{}
\titleformat{\subsubsection}
{\normalfont\small\bfseries}{}{0mm}{}  
\titlespacing{\section}{0pt}{*0.5}{*0.0}
\titlespacing{\subsection}{0pt}{*0.5}{*0.0}
\titlespacing{\subsubsection}{0pt}{*0.5}{*0.0}

% ---- FORMATO DE LISTAS ----
\usepackage{enumitem}
\setlist[itemize]{leftmargin=*, nosep}

\title{}
\pagenumbering{gobble}                                   % Quitar número de la página

% ---- FORMATO DE PÁRRAFOS ----
\setlength{\parindent}{0em}
\setlength{\parskip}{0.4em}

% ---- TEXTO DE RELLENO ----
\usepackage{lipsum}  

\begin{document}

\footnotesize

\begin{center}
     \Large{\textbf{Título del documento}} \\
\end{center}

\begin{multicols*}{3}
\setlength{\premulticols}{1pt}
\setlength{\postmulticols}{1pt}
\setlength{\multicolsep}{1pt}
\setlength{\columnsep}{2pt}

\setlength{\abovedisplayskip}{0.25em}
\setlength{\belowdisplayskip}{0.25em}
\setlength{\abovedisplayshortskip}{0.25em}
\setlength{\belowdisplayshortskip}{0.25em}

%%%%%% TEXTO AQUÍ %%%%%%
% Quitar  lo que está aquí. Nada más es para el ejemplo

\section{Sección 1}
\lipsum[1-3]

\section{Sección 2}
\lipsum[4-6]

\subsection{Subsección 2.1}
\lipsum[7-8]

\[ f_X(x) = \frac{x^{\alpha - 1} e^{-\frac{x}{\beta}}}{\beta^\alpha \Gamma(\alpha)} \mathbb{I}_{\mathbb{R^{+}}}(x) \]

\lipsum[9]

\section{Sección 3}
\lipsum[10-14]

%%%%%% FIN DEL TEXTO %%%%%%

\vfill
\begin{flushright}
    \rule{0.65\linewidth}{0.1pt} \\
    José Alberto Márquez Luján
\end{flushright}

\end{multicols*}

\end{document}